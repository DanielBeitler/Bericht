%-------------------------------------
%-- Informationen für das Deckblatt --
%-------------------------------------
\newcommand{\studiengang}{\textsf{Informationstechnik}}

\newcommand{\titel}{im Studiengang \studiengang\\ an der Dualen Hochschule Baden-Württemberg Mannheim}

\newcommand{\praxisA}{Praxisphase 1: 09. Jänner 2016 - 17. März 2016}
\newcommand{\praxisB}{Praxisphase 2: -/-}

\newcommand{\themaA}{Thema 1:\\-/-}														
\newcommand{\themaB}{Thema 2:\\-/-}

\newcommand{\abgabe}{Abgabe: 30.09.2016\\}


\newcommand{\autor}{Beitler, Daniel}
\newcommand{\matrikelnr}{9247761}
\newcommand{\jahrgang}{TINF16ITIN}

\newcommand{\dlr}{Deutsches Zentrum für\\ &Luft- und Raumfahrt e.V.\\ &in der Helmholtz-Gemeinschaft}
\newcommand{\standort}{Oberpfaffenhofen}

\newcommand{\institut}{Institut für Hochfrequenztechnik und Radarsysteme}
\newcommand{\abteilung}{SAR-Technologie}
\newcommand{\betreuer}{Dr. Rolf Scheiber}

%----------------------------------
%-- Vordefinierte Formatierungen --
%----------------------------------
\newcommand{\equname}{Gleichung}						% \equname		-> Gleichung

% Darstellung von Gleichungen, welche unformatiert (nicht zentriert und ohne Nummerierung) ausgegeben werden:
\newcommand{\EQ}[1]
{ 
	\ensuremath{#1}
}

%---------------------------------------------------------

% Darstellung von Abbildungenen mit...

\newcommand{\FG}[4]
{ 
	\vspace{2mm}
	 
	\begin{figure}[!htb]
	
		\centering
		
			\includegraphics[height = #2]{#1} 	
			\caption{#3}
			\label{pic:#4}
			 
	\end{figure}
}

% ... und Referenz
\newcommand{\FR}[1]{\figurename~\ref{pic:#1}}
%-------------------------------------------------------

% Darstellung von Tabellen mit...

\newcommand{\TB}[4]
{ 
	\vspace{2mm}
	 
	\begin{table}[#1]
	
		\centering
	
			\input{#2} 			 
	
		\caption{#3}
		\label{tab:#4}
		
	\end{table}
}

% ... und Referenz
\newcommand{\TR}[1]{\tablename~\ref{tab:#1}}


% ---------Schriftformatierungen------------------ 
\newcommand{\Name}[1]{\emph{#1}}
\newcommand{\Fachbegriff}[1]{\textbf{#1}} 
\newcommand{\Code}[1]{\texttt{#1}}
\newcommand{\Datei}[1]{\texttt{#1}}
\newcommand{\Datentyp}[1]{\textsf{#1}}