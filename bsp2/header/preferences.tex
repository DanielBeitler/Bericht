	\parskip3pt												% Zeilenabstand nach Absatz
	\parindent6pt											% horizontaler Abstand nach Absatz
	\setkomafont{sectioning}{\scshape\bfseries}       			% serifenbehafftete Übeschriften
	\setkomafont{captionlabel}{\bfseries}     			  		% Fette Beschriftungen für alle "caption"
	\renewcommand{\familydefault}{\sfdefault}					% Verwendung von serifenloser Schrift im gesamten Dokument
	\captionsetup{figurewithin = none, tablewithin = none}		% fortlaufende Nummerierung von Abbildungen und Tabellen
	
	
%---------------------------
%-- Listing-Einstellungen --
%---------------------------

\makeatletter
\AtBeginDocument	
{													
	\renewcommand*{\thelstlisting}{\arabic{chapter}.\arabic{lstlisting}}			
}
\makeatother													

\renewcommand{\lstlistingname}{Quellcode} 

\lstset{
		captionpos 			= b,                   				% Caption-Position unten, weitere Option: t (Caption-Position oben)
		numbers 				= left,            					% Zeilennummern links
		% stepnumber 		= 1,            						% Jede Zeile nummerieren, default: 1
		numbersep 			= 5pt,           					% 5pt Abstand zum Quellcode
		numberstyle 			= \tiny,       						% Zeichengrösse 'tiny' für die Nummern, weitere Option: \footnotesize
		breaklines 			= true,         						% Zeilen umbrechen, wenn notwendig.
		breakautoindent 		= true,    							% Nach dem Zeilenumbruch Zeile einrücken
		% postbreak 			= \space,        					% Bei Leerzeichen umbrechen.
		tabsize 				= 2,               					% Tabulatorgrösse 2
		frame 				= shadowbox,							% Schattenbox, auch möglich: single
		rulesepcolor 		= \color{Gray},						% Farbe des Schattens
		framexleftmargin 	= 5mm,								% linken Rand vergrößern, um Zeilennummerierung mit einzubinden
		basicstyle 			= \ttfamily\footnotesize,				% \footnotesize \small \scriptsize
		keywordstyle 		= \color{Red}\bfseries,				% Schlüsselwortfarbe
		commentstyle 		= \color{ForestGreen}\bfseries, 		% Kommentarfarbe
		stringstyle 			= \color{NavyBlue},					% Farbe der Strings
		showspaces			= false,        						% Leerzeichen nicht anzeigen.
		showstringspaces		= false   							% Leerzeichen auch in Strings nicht anzeigen.
	   }
	
	
%-------------------------
%-- Kopf- und Fußzeile --
%-------------------------
		
\pagestyle{scrheadings}
		
\clearscrheadfoot 												% Kopfzeilen "säubern", um eigene Einstellungen vorzunehmen
		
\automark{chapter} 												% Die Anweisung \automark aktiviert die automatische 
								 								% Aktualisierung des Kolumnentitels. 
		
\setheadsepline{0.5pt}											% Trennlinie in der Kopfzeile
% \setfootsepline{0.5pt}											% Trennlinie in der Fußzeile
		
%-- Kopfzeile --
\ohead{\rightmark}												% Ebene "chapter" rechtsbündig setzen
\setkomafont{pageheadfoot}{\normalfont\sffamily\bfseries\small}  	% Schriftart des Kopfzeiletitels
	
\renewcommand*{\chapterheadstartvskip}{\vspace*{.5\baselineskip}}
	
%-- Seitennummerierung --
\cfoot[\pagemark]{\pagemark}										% Seitennummer zentriert setzen
\setkomafont{pagenumber}{\normalfont\sffamily}						% Schriftart des Fußzeiletitels

%-----------------------------------------------
%-- Optionen für Links und pdf Informationen --
%-----------------------------------------------
	
\hypersetup{					
		pdftitle				= 	{Praxisarbeit - \autor},   						% Titel der Arbeit
		pdfauthor			= 	{\autor},     					% Autor, in "makro.sty" festgelegt
		pdfsubject			=	{Praxisarbeit - \autor},   						% Betreff der Arbeit, in "makro.sty" festgelegt				
		bookmarksnumbered	= 	true, 							% Nummerierung der Kapitel in Lesezeichenleiste
		colorlinks			= 	true,       						% sollen Links farbig gemacht werden? 
		linkcolor			= 	Black,          					% Links (Inhaltsverzeichnis, Abbildungen, Tabellen usw.)
		citecolor			= 	Black,        					% Links zum Literaturverzeichnis
		urlcolor				= 	Brown,    						% Farbe der URL
		filecolor			= 	Brown							% kann auskommentiert werden, wenn keine Verlinkung auf
																% externe Dateien verwendet wird
		}