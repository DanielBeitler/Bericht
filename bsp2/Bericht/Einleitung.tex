\chapter{Einleitung}
\section{Das DLR}
TODO: - Ändern?

Das Deutsche Zentrum für Luft- und Raumfahrt e.V. ist eine Forschungseinrichtung im Auftrag des Bundes. Es ist in vielen Forschungsbereichen der Luft- und Raumfahrt, aber auch Verkehrstechnik vertreten. Aufgeteilt ist das DLR in 33 verschiedene Institute, die in jeweils unterschiedlichen Bereichen tätig sind. Das DLR ist an 16 Standorten vertreten und beschäftigt ca. 8000 Mitarbeiter\cite{dlr}.

\subsection{Institut für Hochfrequenztechnik und Radarsysteme}
Das Institut für Hochfrequenztechnik und Radarsysteme entwickelt Systeme für die Fernerkundung mittels passiven und aktiven Mikrowellen.\\
Aufgaben des Instituts sind die Umsetzung von Konzepten in der flugzeug- und satellitengestützten Fernerkundung sowie im Bereich des Verkehrsmanagement und der Aufklärung und Sicherheit.\\
Im Rahmen verschiedener, auch langfristiger (Forschungs-)Programme kooperiert das Institut eng mit dem Raumfahrt-Management des DLR in Bonn, der Industrie, verschiedenen Ministerien und der European Space Agency (ESA), dem Raumfahrprogramm Europas

\subsection{Abteilung SAR-Technologie}
Aufgabe der Abteilung SAR Technologie ist die Entwicklung von Techniken, die sich mit Fokussierungs- und Bildanalysealgorithmen von Radarbildern und dem Betrieb von flugzeuggestützten Radarsystemen befassen.\\
Hauptaufgabe hierbei ist die Entwicklung neuer Radartechniken und die Demonstration derselben, da flugzeuggestützte Radarsysteme flexibel und sehr einfach zu konfigurieren sind.

