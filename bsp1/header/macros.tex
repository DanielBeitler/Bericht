%-------------------------------------
%-- Informationen für das Deckblatt --
%-------------------------------------
\newcommand{\studiengang}{\Large\sf Studiengang Informationstechnik}

\newcommand{\titel}{Bericht zum Modul Praxis I}

\newcommand{\praxisA}{Praxisphase 1: 10. Januar 2011 - 25. März 2011\\}


\newcommand{\themaA}{Thema 1:}														
% \newcommand{\themaB}{Thema 2: }

\newcommand{\autor}{\bf Willy Wiener}
\newcommand{\matrikelnr}{\it - Matrikelnr.: 12345678 -}
\newcommand{\jahrgang}{\it - Kurs: TITXXZ -}

\newcommand{\dlr}{Deutsches Zentrum für\\ Luft- und Raumfahrt e.V.\\ in der Helmholtz-Gemeinschaft}
\newcommand{\standort}{Göttingen}

\newcommand{\institut}{Institut für Aerodynamik und Strömungstechnik}
\newcommand{\abteilung}{Abteilung: Raumfahrzeuge}
\newcommand{\betreuer}{Foo Bar}

%----------------------------------
%-- Vordefinierte Formatierungen --
%----------------------------------
\newcommand{\equname}{Gleichung}						% \equname		-> Gleichung

% Darstellung von Gleichungen, welche unformatiert (nicht zentriert und ohne Nummerierung) ausgegeben werden:
\newcommand{\EQ}[1]
{ 
	\ensuremath{#1}
}

%---------------------------------------------------------

%format: \setPicture{path}{height in cm}{description}{label for references}
\newcommand{\setPicture}[4]{
	\begin{figure}[!h]
		\centering
		\includegraphics[height=#2cm]{#1}
		\caption{#3}
		\label{#4}
	\end{figure}
}

%format: \setPictureSourced{path}{height in cm}{description}{label for references}{source}
\newcommand{\setPictureSourced}[5]{
	\begin{figure}[!h]
		\centering
		\includegraphics[height=#2cm]{#1}
		\caption[#3]{#3\protect\footnotemark}
		\label{#4}
	\end{figure}
	\footnotetext{Quelle: #5}
}

%format: \refPicture{label for references}
\newcommand{\refPicture}[1]{\figurename~\ref{#1}}

% Darstellung von Abbildungenen mit...

\newcommand{\FG}[4]{ 
	\vspace{2mm} 
	\begin{figure}[!htb]
		\centering
		\includegraphics[height = #2]{#1} 	
		\caption{#3}
		\label{pic:#4}		 
	\end{figure}
}

% ... und Referenz
\newcommand{\FR}[1]{\figurename~\ref{pic:#1}}
%-------------------------------------------------------

% Darstellung von Tabellen mit...

\newcommand{\TB}[4]
{ 
	\vspace{2mm}
	 
	\begin{table}[#1]
	
		\centering
	
			\input{#2} 			 
	
		\caption{#3}
		\label{tab:#4}
		
	\end{table}
}

% ... und Referenz
\newcommand{\TR}[1]{\tablename~\ref{tab:#1}}


% ---------Schriftformatierungen------------------ 
\newcommand{\Name}[1]{\emph{#1}}
\newcommand{\Fachbegriff}[1]{\textbf{#1}} 
\newcommand{\Code}[1]{\texttt{#1}}
\newcommand{\Datei}[1]{\texttt{#1}}
\newcommand{\Datentyp}[1]{\textsf{#1}}